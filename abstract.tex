\begin{abstract}
%70-150words
Challenge-response (CR) is an effective way to authenticate users even if the communication channel is insecure. Traditionally, CR authentication relies on one-way hashes and shared secrets to verify the identities of users. Such a method cannot cope with an insider attack, where a user obtained the secret from a legitimate user. To cope with it, we design a biometric-based CR authentication scheme (hereafter \CiT), which is derived from the motions as a user operates emerging depth-sensor-based input devices, such as a Leap Motion controller. We envision that to authenticate a user, \CiT randomly chooses a string (e.g., a few words), and the user has to write the string in the air. Utilizing Leap Motion, \CiT captures the user's writing movements and then extracts his handwriting style. After verifying that what the user writes matches what is asked for, \CiT leverages a Support Vector Machine (SVM) with co-occurrence matrices to model the handwriting styles and can correctly authenticate users, even if what they write is completely different every time. Evaluated on data from 24 subjects over 7 months, \CiT managed to verify a user with an average of $1.18\%$ (Equal Error Rate) EER and to reject imposers  with $2.26\%$ EER.



%\keywords{Authentication, challenge-response, biometrics, SVM, handwriting style, leap motion}

\end{abstract}